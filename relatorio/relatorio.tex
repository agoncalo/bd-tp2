\documentclass[10pt,a4paper,oneside,titlepage]{article}
\usepackage[utf8]{inputenc}
\usepackage[portuguese]{babel}
\usepackage[T1]{fontenc}
\usepackage{lmodern}
\usepackage{amsmath}
\usepackage{amsfonts}
\usepackage{amssymb}
\usepackage{graphicx}
\usepackage{lmodern}
\author{Antônio M. Filho \\ Christian Reis \\ Daniel Carvalho \\ Rodrigo Carvalho}
\title{RELATÓRIO REFERENTE AOS MÉTODOS DE EXTRAÇÃO E ARMAZENAMENTO DE CONTEÚDO DE NOTÍCIAS DISPONÍVEIS NA WEB}
\date{Novembro de 2017}

\begin{document}

\maketitle

\section{Introdução}

Foi proposto a busca e extração de dados disponíveis na Web com o conteúdo de notícias que abrangem diversos temas. Com estes dados é possível retirar informações acerca do conteúdo, podendo categorizá-los e classificá-los de acordo com determinados critérios. Para realizar-se a busca destes dados, é utilizado como ferramenta principal um \textit{crawler} para fazer buscas em sites de notícias e extrair o conteúdo, e o armazenamento deste conteúdo é feito na núvem. Este relatório destaca as ferramentas utilizadas para esta tarefa inicial.

\section{Utilização do \textit{Crawler}}



\section{Armazenamento na Nuvem}



\section{Resultados Parciais e Conclusão}



\end{document}
